
\secly{Язык \bi}

\begin{itemize}[nosep]
\item DSL-ориентированный
\item динамический
\item объектный
\item параллельный
\end{itemize}

язык программирования и проектирования программно-аппаратных систем.

\bigskip
Язык \bi\ был создан в экспериментах с синтаксическим анализом, символьными
вычислениями, и чтении книг типа \cite{sicp,dragon,st}. Движущей силой было
желание создать \emph{инструмент для метапрограммирования: описания прикладных 
программ и программно-аппаратных систем в свободном синтаксисе}, с применением
смешанных методов программирования, и без необходимости полностью описывать все 
объекты программы во всех тонкостях и с соблюдением всех заморочек языка
реализации, библиотек, платформ и ОС.

Идея о свободном синтаксисе предполагала, что программист имеет полный контроль
над языком, так как сам написал его реализацию, и по мере необходимости меняет 
работу существующего функционала языковой системы, и дописывает дополнительные
фичи. В процессе работы над различными проектами в итоге формируется некий клон
базового языка, заточенный под решаемые задачи, под используемые платформы, и с 
определенным набором средств автоматизации рутинных вопросов, типа генерации 
кода на Си, автоматического отслеживания зависимостей между файлами и т.п.

