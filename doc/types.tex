
\secrel{Система типов}\secdown

\secrel{Нативные}\secdown

Нативные типы реализуются на уровне \bivm\note{виртуальная машина, 
рантайм-движок языковой системы \bi}, 
являются типами самого низкого уровня, и не покрываются в полной мере 
средствами отладки, трассировки, и контроля. При кросс-трансляции преобразуются
в базовые типы целевого языка, и элементы данных с аппаратной 
поддержкой\note{машинное слово, блок памяти, элемент стека мат.сопроцессора и 
т.п.}.

Особенно аккуратно их нужно использовать при трансляции для 8/16-битных 
платформ, так как низкая разрядность нативных типов и облегченный контроль со 
стороны \bi-машины дает весь спектр побочных эффектов: переполнения, знаковые 
ошибки, потерю точности и т.п. 

\secrel{\class{c}: одиночный символ /\emph{[c]har}/}
\secrel{\class{s}: строка /\emph{[s]tring}/}
\secrel{\class{i}: целое /\emph{[i]nt}/}
\secrel{\class{f}: число с плавающей точкой /\emph{[f]loat}/}

\secup

\secup
