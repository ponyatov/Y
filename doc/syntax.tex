\secrel{Синтаксис языкового ядра}\secdown

\begin{framed}\noindent
\bi\ построен на bypass-принципе: все, что языковая система получила на вход, и
не распознала как элемент языка, отправляется на выход без изменений.
\end{framed}

Также используется литературное программирование\note{literate programming}: 
решение задачи описывается на удобном языке разметки (\LaTeX, DocBook,..), 
а \emph{\bi\ вычленяет и выполняет блоки кода}. 
Поэтому комментарии в программах на \bi\ \emph{не используются}.

\begin{framed}\noindent
Для упрощения отладки и разработки "ядерная"\ реализация языка \biz\ выдает на 
stdout html-заголовок, позволяет использовать в исходном тексте html-тэги, 
и при вычислении выражений результат также обрамляется в стилевые тэги в 
соответствии с типом полученного значения.

Использование на начальном этапе разработки языка такой 
"обрезанной"\ html-разметки вместо полноценного документирования в формате 
\LaTeX\ позволяет исключить использование относительно медленного
\TeX-процессора для просмотра результата: используется быстрый минибраузер, 
встроенный в IDE \eclipse. Кроме того, этот подход оказывается удобным и для 
встраиваемых систем с веб-интерфейсом. 
\end{framed}

\secly{Комментарии}

Тем не менее из условий запуска скриптов в UNIX требуется, чтобы первая строка
имела формат \verb#|!/bin/bI|#,
поэтому в язык все же вводятся лексемы комментариев:

\begin{verbatim}
#!/bin/bI
# строчный комментарий
#| блочный 
             комментарий |#
\end{verbatim}

\secly{Точечные команды}

Точечные dot-команды используются для документирования и управления форматами 
вывода \bi-системы\note{точка была взята из языка \forth, подразумевает 
конструкции языка, относящиеся к выводу}.

\bigskip
\begin{tabular}{l l}
\hline
.html & режим html-форматирования, \emph{для \biz\ единственный поддерживаемый
режим} \\
.tex & \LaTeX-разметка \\
\hline
.sec & секция документации \\
.sec- .sec+ & изменение текущего уровня секционирвания (глава, раздел,..) \\
\hline
\end{tabular}

\secup
