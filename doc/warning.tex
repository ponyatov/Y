\secly{\textsc{Be Warned}}\label{warning}

\subsection{\bi\ not intended for data crunching itself}

\begin{framed}
\noindent \textcolor{Blue}{\bi\ not intended for data crunching itself\ --- it's
tool for hand-cranked compiling and program transformations.}\\\bi\ core supports
\verb|<num:|\ref{num}\verb|>|\
data type for floating point numbers, but \emph{avoid use of
\bi\ core for numerical computation}.
\textcolor{Green}{Right way to use \bi\ --- construct low-level program
which will crunch your data using power of \bi\ metaprogramming.}

\figx{tmp/meta.png}{width=0.9\textwidth}

\end{framed}

LLVM framework and JIT libraries looks very interesting for
\term{dynamic compilation}\ --- this magic can conjure some speedup of
\bi\ core\note{it's high-level part realized in \bi\ language, and \bi/$next$
generation described via core metamodel}\ itself,
and incredible performance of mutable runtime-generated machine code 
for data crunching.    

\pagebreak
\subsection{There is no memory management at all}

\begin{framed}\begin{framed}\noindent \textcolor{Red}{Current 
version of \bi\ core have no any memory management: there is no garbage
collector, all created objects will be stay in memory until system crash on
memory overflow.}
\end{framed}
This way was chosen for simplicity. It is sufficient for tiny batch runs
and interactive work with "failure and restart from snapshot"\ hints,
but this makes continues or large data crunching impossible.
\end{framed}

\subsection{You must have some skills in compiler design and functional programming}

\begin{framed}
\noindent \bi\ system is syntax analyzer and translator framework by design, 
and user must have some skills in compiler design and functional programming.
You must read DragonBook \cite{dragon}, SICP \cite{sicp}\ and
Harrison/Field \cite{hf}\ before you dig in hedgehog den.  
\end{framed}