%%%%%%%%%%%%%% HEADER %%%%%%%%%%%%%%%%% 

% multivolume e-book
\documentclass[oneside,12pt]{book}
\usepackage[paperwidth=210mm,paperheight=148mm,margin=5mm]{geometry}

% font setup for screen reading
\renewcommand{\familydefault}{\sfdefault}
\normalfont

% pdflatex options
\usepackage[unicode]{hyperref}
\newcommand{\email}[1]{$<$\href{mailto:#1}{#1}$>$}
\usepackage[pdftex]{graphicx}

% relative sectioning
\usepackage{ifthen}
\newcounter{secdepth}\setcounter{secdepth}{0}
\newcommand{\secup}{\addtocounter{secdepth}{1}}
\newcommand{\secdown}{\addtocounter{secdepth}{-1}}
\newcommand{\secrel}[1]{
\ifthenelse{\equal{\value{secdepth}}{0}}{\part{#1}}{}
\ifthenelse{\equal{\value{secdepth}}{-1}}{\chapter{#1}}{}
\ifthenelse{\equal{\value{secdepth}}{-2}}{\section{#1}}{}
\ifthenelse{\equal{\value{secdepth}}{-3}}{\subsection{#1}}{}
\ifthenelse{\equal{\value{secdepth}}{-4}}{\subsubsection{#1}}{}
}
\newcommand{\secly}[1]{
\section*{#1}
\addcontentsline{toc}{section}{#1}
}

% colors
\usepackage{xcolor}

% listings
\usepackage{verbatim}
\usepackage{listings}
\lstset{
basicstyle=\small, % or \tiny \small or \footnotesize
extendedchars=true,inputencoding=utf8, % i18n
frame=single, % show frames around
numbers=left, numberstyle=\small,numbersep=1mm,% line numbering
tabsize=4, % tab style
keywordstyle=\color{blue},%\texttt,
keywordstyle={[2]\color{green}},%\texttt,
keywordstyle={[3]\color{magenta}},%\texttt,
keywordstyle={[4]\color{red}},%\texttt,
keywordstyle={[5]\color{blue}},%\texttt,
commentstyle=\color{cyan}%\texttt%,
% showspaces=false
}
\newcommand{\lst}[3]{\lstinputlisting[title=\href{#2}{#1}]{#3}}
\newcommand{\lstx}[4]{\lstinputlisting[title=\href{#2}{#1},language=#4]{#3}}

\newcommand{\file}[1]{\textcolor{magenta}{#1}}


% misc
\usepackage{enumitem}	% nosep option in lists/enums

\newcommand{\note}[1]{\footnote{\ #1}}

\newcommand{\cpp}{\textcolor{blue}{$C^+_+$}}
\newcommand{\py}{\textcolor{blue}{$Python$}}
\newcommand{\st}{\textcolor{blue}{$SmallTalk$}}
\newcommand{\lisp}{\textcolor{blue}{$LISP$}}
\newcommand{\bi}{\textcolor{blue}{$bI$}}

\newcommand{\class}[1]{\textcolor{blue}{#1}}

% Cyrillization
\usepackage[T1,T2A]{fontenc}
\usepackage[utf8]{inputenc}
\usepackage[english,russian]{babel}
\usepackage{indentfirst}

%\newcommand{\bi}{\textcolor{blue}{\textsc{Ы}}}

%%%%%%%%%%%%%%%%%%%% MANUAL %%%%%%%%%%%%%%%%%%%%%%%%% 

\title{Язык \bi}
\author{\copyright\ Dmitry Ponyatov \email{dponyatov@gmail.com}}
\begin{document}

\maketitle

\tableofcontents

\secly{Язык \bi}

\begin{itemize}

\item динамический
\\ в \lisp-смысле: программа сама является структурой данных, и может
модифицировать другие программы или сама себя, создавать и удалять программы в
процессе выполнения.

\item DSL-ориентированный
\\ так как очень часто приходиться работать с данными в
текстовых форматах, в ядре языка предусмотрен функционал создания парсеров, 
оптимизаторов, трансляторов и компиляторов любых других языков.

\item объектный
\\ в стиле \st.

\item параллельный
\\ при разработке языка большое внимание уделяется обеспечению параллелизма в 
самом широком смысле: микропотоки, использование потоков runtime-системы и/или
ОС, управление выполнением программ на гетерогенных вычислительных кластерах, 
облачных севисах и p2p распределенных сетях, средства платформенно-независимой
сериализаци, поддержка персистентности и резервирования, синтаксическая 
поддержка параллельного программирования.

\item run-time спецификация
\\ объектов в процессе выполнения программы не требует обязательное 
предварительное определение объектов, при попытке использования несуществующего 
объекта открывается интерактивная отладочная сессия.

\item интерактивная отладка
\\ в стиле SmallTalk позволяет программисту создавать программу в диалоговом
режиме в процессе отладки и прогона на тестовом стенде или наборе юнит-тестов.

\item компилируемый через трансляцию
\\ результирующая система может быть оттранслирована\note{ограниченно}\ в 
исходный код на \cpp\note{и \py}\ для обеспечения переносимости программ для систем,
для которых не реализована \bi-машина, недостаточно аппаратных 
ресурсов\note{встраиваемые системы}, или предъявляются жесткие требования по 
надежности\note{$\uparrow$}.

\item литературное программирование\note{literate programming}\ и 
автоматическая генерация документации  

\end{itemize}

\secrel{Синтаксис}\secdown

\secrel{Комментарии}

\secly{строчный}

\begin{verbatim}
// line comment //
\end{verbatim}

\secly{блочный}

\begin{verbatim}
\\ block
        comment \\
\end{verbatim}

\secly{разметка для документации}

\secrel{Литералы}

\secly{числа}

\secly{строки}

\secrel{Контейнеры}

\secly{пары}

\secly{списки}

\secly{словари}

\secup

\secrel{Реализация}\secdown

\secrel{Ядро на \cpp}\secdown

\secrel{Файлы}

\begin{tabular}{l l}
Makefile & файл проекта \\
\end{tabular}

\secrel{\file{lexer.lpp}: лексер}
\secrel{\file{parser.ypp}: парсер}
\secrel{\file{core.cpp}: runtime ядро}
\secrel{\file{bI.h}: декларации типов}

\secup

\secup

\secrel{Система типов}\secdown

\secrel{Базовые}\secdown

\secrel{\class{object}: объект}

\secrel{\class{int}: целое число}

\secrel{\class{float}: число с плавающей точкой}

\secrel{\class{string}: строка}

\secup

\secrel{Контейнеры}\secdown
\secrel{\class{pair}: пара}
\secrel{\class{list}: список}
\secrel{\class{dict}: словарь}
\secrel{\class{class}: класс}
\secup

\secrel{Файлы}\secdown
\secrel{\class{file}: бинарный файл /binary/}
\secrel{\class{text}: текстовый файл /plain text/}
\secrel{\class{lexer}: лексер}
\secrel{\class{parser}: парсер синтаксиса}
\secrel{\class{data}: синтаксически размеченный файл с данными}
\secup

\secrel{Многозадачность}
\secrel{GUI}
\secrel{Сетевой функционал}
\secrel{Мультимедиа}
\secrel{Математические}
\secrel{Геометрия и САПР}

\secup

\end{document}
