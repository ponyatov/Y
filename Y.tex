%%%%%%%%%%%%%% HEADER %%%%%%%%%%%%%%%%% 

% multivolume e-book
\documentclass[oneside,12pt]{book}
\usepackage[paperwidth=210mm,paperheight=148mm,margin=5mm]{geometry}

% font setup for screen reading
\renewcommand{\familydefault}{\sfdefault}
\normalfont

% pdflatex options
\usepackage[unicode,colorlinks=true]{hyperref}
\newcommand{\email}[1]{$<$\href{mailto:#1}{#1}$>$}
\usepackage[pdftex]{graphicx}

% relative sectioning
\usepackage{ifthen}
\newcounter{secdepth}\setcounter{secdepth}{0}
\newcommand{\secup}{\addtocounter{secdepth}{1}}
\newcommand{\secdown}{\addtocounter{secdepth}{-1}}
\newcommand{\secrel}[1]{
\ifthenelse{\equal{\value{secdepth}}{0}}{\part{#1}}{}
\ifthenelse{\equal{\value{secdepth}}{-1}}{\chapter{#1}}{}
\ifthenelse{\equal{\value{secdepth}}{-2}}{\section{#1}}{}
\ifthenelse{\equal{\value{secdepth}}{-3}}{\subsection{#1}}{}
\ifthenelse{\equal{\value{secdepth}}{-4}}{\subsubsection{#1}}{}
}
\newcommand{\secly}[1]{
\section*{#1}
\addcontentsline{toc}{section}{#1}
}

% colors
\usepackage{xcolor}

% listings
\usepackage{verbatim}
\usepackage{listings}
\lstset{
basicstyle=\small, % or \tiny \small or \footnotesize
extendedchars=true,inputencoding=utf8, % i18n
frame=single, % show frames around
numbers=left, numberstyle=\small,numbersep=1mm,% line numbering
tabsize=4, % tab style
keywordstyle=\textsc
}
\newcommand{\lst}[2]{\lstinputlisting[title={#1}]{#2}}
\newcommand{\lstx}[3]{\lstinputlisting[title={#1},language=#3]{#2}}
\newcommand{\file}[1]{\texttt{#1}}
\newcommand{\class}[1]{\textcolor{blue}{#1}}

% programming languages
\newcommand{\cpp}{$C^+_+$}
\newcommand{\py}{$Python$}
\newcommand{\st}{$SmallTalk$}
\newcommand{\lisp}{$LISP$}
\newcommand{\bi}{$bI$}
\newcommand{\bivm}{$bI_{VM}$}
\newcommand{\eclipse}{\textcircled{$\equiv$}\textsc{eclipse}}

% misc
\usepackage{enumitem}	% nosep option in lists/enums

\newcommand{\note}[1]{\footnote{\ #1}}
\renewcommand{\emph}[1]{\textcolor{blue}{#1}}

% Cyrillization
\usepackage[T1,T2A]{fontenc}
\usepackage[utf8]{inputenc}
\usepackage[english,russian]{babel}
\usepackage{indentfirst}

\input{../texheader/cyr}
\input{../texheader/comp}

\author{Dmitry Ponyatov \email{dponyatov@gmail.com}}
\title{Язык Ы}

\begin{document}
\input{../texheader/title}

\tableofcontents

\secdown

\secrel{Задачи разработки}

\begin{itemize}[nosep]
  \item чтобы никто не догадался
  \item по мотивам SmallTalk: динамический язык сверх-высокого уровня
  \item эксперименты с программированием трансляторов, компиляторов и DSL
\end{itemize}

\secrel{Архитектура}\secdown

\secrel{Входной поток (исходник) и пасер/интерпретатор}

Интерпретатор входного потока читает стандартный ввод (stdin) или текстовые
файлы через \verb|include|. Чтение выполняется посимвольно до
символа-разделителя\ --- пробела или конца строки. Выделенный блок символов 
до разделителя: \emph{слово}\ ищется в \emph{словаре} \verb|World|\ и
выполняется как функция.

\secrel{Стек Stack}\secdown

Для хранения промежуточных данных используется рабочий стек.

\secrel{. (точка)}

Точка сбрасывает рабочий стек. Ее нужно применять в конце каждого предложения
или определения.

\secrel{?}

Вывод сдержимого стека.

\secup

\secrel{Образ World}

\secup

\secrel{Синтаксис}\secdown

\secrel{Комментарии}

Любой текст после символа \#\ до конца строки игнорируется.
Выбор символа определен необходимостью его использования в начале файла
скриптов под UNIX в виде \verb|#!/usr/bin/Y|.

\secrel{Управление системой и образом}\secdown

\secrel{end}

\secrel{dump}

\secup

\secup

\secrel{Реализация}

\begin{itemize}[nosep]
  \item \file{Makefile}
  \item \file{Y.py}
  \item \file{system.Y}
\end{itemize}

\end{document}
